%! TeX Program = LuaLaTeX

\makeatletter
\def\ltj@stdmcfont{SourceHanSerifSC}
\def\ltj@stdgtfont{SourceHanSansSC}
\def\ltj@stdyokojfm{eva/{nstd,smpl,hgp}}
\def\ltj@stdtatejfm{eva/{nstd,smpl,hgp,vert}}
\makeatother

\documentclass[twoside]{ltjsarticle}

%%\usepackage[deluxe]{luatexja-preset}

\usepackage{luatexja-fontspec}
\setmainfont{Linux Libertine O}
\setsansfont{Linux Biolinum O}
\setmonofont[Scale=MatchLowercase]{CMU Concrete}

\usepackage{listings}
\lstset{
  basicstyle=\small\ttfamily,
  breaklines=true,
  columns=fullflexible,
  numbers=left,
  numberstyle=\tiny,
  stepnumber=1,
  gobble=2,
  numbersep=6pt,
  escapechar = §
}
\def\meta#1{{\small\normalfont\rmfamily\itshape$\langle$#1\/$\rangle$}}

\usepackage{graphicx}
\def\jstml{{\ltjsetparameter{yjabaselineshift=0pt, yalbaselineshift=-.18ex}%
    \reflectbox{\fontspec[AutoFakeBold=0.2]{Linux Libertine O}J}%
    \raise.3ex\hbox{\kern-.04em{\small\sffamily\scshape{stm}}}%
    \kern-.22em\lower.45ex\hbox{\scalebox{-1}[1.26]{\fontspec[AutoFakeBold=0.2]{Linux Libertine O}L}}}}

\begin{document}

\parindent=2\zw

\title{\bfseries{\jstml}\jfontspec{HaranoAjiMincho-SemiBold.otf}\,\,文\,本\,結\,構\,化\,標\,記\,語\,言\\[2pt]%
       {{\normalsize\mdseries\itshape (a.k.a.,)\ \ \ }\Large Jing's Structured Text Markup Language}}
\author{黄{\quad}京 (\textit{RadioNoiseE})}
\date{西\,历\today}
\maketitle

\begin{abstract}
  本文档将介绍{\scshape JSTML},一种基于{\scshape C}语言构建的(极简易的)文本结构化标记语言;而它的设计目的是,用来写同学录。\par
  其本质上是一个基于下推自动机\footnote{一个很哲学的术语。}的解析器,因设计用途的局限性,不允许出现嵌套等魔法。
  容错模型也较为简陋,而性能则没有进行任何优化(读入输出纯靠栈)。\par
  将先介绍数据结构、语法,而后介绍实现细节、自动化{\scshape Lua}脚本、{\scshape plain-TeX}输出样式文件等信息。
\end{abstract}

\section{一些约定}
\begin{itemize}
  \item 下文中将会用「她」「其」等代词表示{\scshape JSTML}语言的独立解释器,即她的可执行文件。
  \item 将会用小型大写西文字母({\scshape Small Caps})表示脚本语言、宏语言、标记语言等,还用来表示某些特定的操作系统名称。
  \item 使用方全角引号(「」)表示被它们划定的特定字符;使用尖括号与斜体表示可选项描述或不需要显示输入的内容,
        不代表实际键入的字符。
  \item 将会使用脚注补充一些多馀信息,且响应国家倡议:正文中使用符合现行语言标准的简化字和两个全角空格宽的缩进。
\end{itemize}

\section{字类型及数据类型、结构}
\subsection{字类型}
在她眼中,所有的输入都属于字(token),及一个或一些字符(character)的集合。
而字又被分为五种类别\footnote{致敬高德纳教授所创{\TeX}的类别码(category code),
由于大部分(如果不是全部的话)标记语言都有类别码的概念、而{\TeX}实际上是宏语言,故特此说明。}:
\begin{description}
  \item[分界符] 如其名,自然是作为两种数据结构的界定出现。其中,「\texttt{<}」被用作表示开始、而「\texttt{>}」表示结束。
  \item[标示符] 标示其中一种数据结构的开始与结束。其中,「\texttt{*}」为开,而「\texttt{/}」为关。
  \item[汉字] 主要的处理对象,也就是这门语言「标记」的东西。由几乎所以不属于其它三类的字符组成。
  \item[空白] 包括空格\footnote{不包括中文的全角空格,其属于汉字类别。}及横向制表符(tab)。
  \item[换行] 指在{\scshape Macintosh}或{\scshape Unix}、{\scshape Posix}等系统下的回车换行符。
\end{description}\par
\subsection{数据类型}
上述的四种字组合便有了能够被她处理的,合法的唯二的数据类型:狗牌(tag)和八卦(text)。\par
其中狗牌表示对八卦的一个概述,故理论上应短小而精悍。也因此,她内部分配给狗牌的空间仅有19个字符长度
\footnote{实现使用\texttt{<uchar.h>}的标准化头文件来支持万国码,所以一个字符是8字节长度。}。
也就是说,如果你往狗牌里硬塞超过19个汉字,会导致分段错误或栈溢出\footnote{这些都是可以调整的,
见\texttt{jstml.c}文件中对\texttt{MAXTOKEN}、\texttt{MAXTAG}等的宏定义。},报错并继续运行
(毕竟它只是狗牌呐)。\par
而另一种八卦所能容纳的字符就多多了,达8192个字符的长度。
八卦与狗牌一一对应,是对狗牌的展开说明,也可以是扩充等一切合理(或不合理)的用途。
\subsection{数据结构}
在两种数据类型的基础上,又构建了两种数据结构(即,用来组织/表示数据类型的东西):片段和累牍。
一个片段只能出现在一行之中\footnote{实际上,是由于它的末尾被且仅被换行符界定。这由于平台差异会出现事故:
{\scshape Windows}下换行符为\texttt{<CR><LF>}、而{\scshape Macintosh}和{\scshape Unix}系为\texttt{<CR>}、
{\scshape Posix}等不明确。故不一定支持使用{\scshape Windows}系统构建项目。},而累牍则理论上横跨数行。\par
片段的狗牌被使用一对分界符界定,而八卦则被结束分界符和换行符界定,语法如下:
\begin{lstlisting}
  §\meta{Optional Space}§<§\meta{Optional Space}§狗牌§\meta{Optional Space}§>§\meta{Optional Space}§八卦§\meta{Carriage Return}§
\end{lstlisting}
而\meta{Optional Space}表示可选的被忽略的空白字类型、\meta{Carriage Return}表示换行符(回车)。\par
累牍的狗牌需被括在分界符中,作为八卦的界定出现两次;同时需要使用累牍标示开、关来标识。语法如下:
\begin{lstlisting}
  §\meta{Optional Space}§<§\meta{Optional Space}§*§\meta{Optional Space}§狗牌§\meta{Optional Space}§>§\meta{Optional Space / Newline}§
    §\meta{Optional Space}§八卦§\meta{Carriage Return}§
    §\meta{Ditto, Iteration / Recurse}§
  §\meta{Optional Space}§<§\meta{Optional Space}§/§\meta{Optional Space}§狗牌§\meta{Optional Space}§>
\end{lstlisting}
其中,\meta{Optional Space / Newline}表示可选的空白类字符、\meta{Ditto, Iteration / Recurse}代表对上一条语句的不限次数的重复
\footnote{当然是在输入的字符不超过那个栈的能力范围的情况下。}。

\section{参考范例}
给出一个实际使用本标记语言的范例(仅供参考、雷同巧合)
\footnote{其中部分文字来源于黄新刚的『雷太赫排版系统简介』,被{\scshape GNU}许可证保护。}:
\begin{lstlisting}
  <姓名> 佚名
  <性别> 不明确,TeX里的\empty、Lua里的nil、C里的0。
  <政治面貌> 革命群众
  <*教育背景>
    巴灵顿大学:烈士(工商管理)——1927~1936
    克莱登大学:勇士(比较文学)——1921~1927
    卧龙岗大学:壮士(分子生物)——1919~1921
    清华学堂:博士(有机化学)——1911~1919
    京师大学堂:硕士(天体物理)——1898~1900
    北洋大学:学士(应用数学)——1895~1898
  </教育背景>
  <*业余爱好>
    搬砖砌墙,割草喂猪;挖坑灌水,淫湿作画。
    研经修佛,以目窥密;布施洗礼,濯尘安卧。
  </业余爱好>
  <*信>
    最大之乘,最正之宗;自如之理,乃见真实;修无为福,
    胜于布施;受持此经,功德无量;应现设化,亦非真实。
  </信>
  <注> 午休时间谢绝来电。
\end{lstlisting}

\section{{\scshape plain-TeX}样式文件使用指北}
本文件就是一个宏集兼驱动文件。
使用它可以将本标记语言输出的中间文件转为PDF文件输出,
通过修改它还可以有不同样式的结果。
默认的标题名非常蠢,强烈建议修改标题后使用(见\texttt{jsdvr.tex}文件中的注释)。\par
另,请注意它是用{\scshape plain-TeX}而非{\LaTeXe}写的,所以别乱改,会出问题。
也请使用{Lua\kern-.14em\TeX}编译。

\section{{\scshape Lua}自动化脚本}
该自动化脚本供处理大批量的使用本标记语言写成的文本文件使用。
只需要创建一个名为\texttt{jsindex.ind}的文本文件,并将需要处理的文件名(带扩展名)
使用西文逗号/换行符分隔后写在该文件里。\par
随后就可以使用{\scshape Lua}的独立解释器执行名为\texttt{jstml-auto.lua}的文件了。
如果一切正常,幸运与你常在的话等一会就会凭空多出一个PDF格式的文件,这就是最终的输出了。
如果报错,能自己解决最好,不行就开issue。

\end{document}
