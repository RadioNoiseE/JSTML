%! TeX Program = LuaLaTeX

\makeatletter
\def\ltj@stdmcfont{SourceHanSerifSC}
\def\ltj@stdgtfont{SourceHanSansSC}
\def\ltj@stdyokojfm{eva/{nstd,smpl}}
\def\ltj@stdtatejfm{eva/{nstd,smpl,vert}}
\makeatother

\documentclass[twoside]{ltjsarticle}

%%\usepackage[deluxe]{luatexja-preset}

\usepackage{luatexja-fontspec}
\setmainfont{Linux Libertine O}
\setsansfont{Linux Biolinum O}
\setmonofont{CMU Concrete}

\usepackage{listings}
\lstset{
  basicstyle=\small\ttfamily,
  breaklines=true
}

\usepackage{graphicx}
\def\jstml{{\ltjsetparameter{yjabaselineshift=0pt, yalbaselineshift=-.18ex}%
    \reflectbox{\fontspec[AutoFakeBold=0.2]{Linux Libertine O}J}%
    \raise.3ex\hbox{\kern-.04em{\small\sffamily\scshape{stm}}}%
    \kern-.22em\lower.45ex\hbox{\scalebox{-1}[1.26]{\fontspec[AutoFakeBold=0.2]{Linux Libertine O}L}}}}

\begin{document}

\parindent=2\zw

\title{\bfseries{\jstml}\jfontspec{HaranoAjiMincho-SemiBold.otf}\,\,文\,本\,結\,構\,化\,標\,記\,語\,言\\[2pt]%
       {{\normalsize\mdseries\itshape (a.k.a.,)\ \ \ }\Large Jing's Structured Markup Language}}
\author{黄{\quad}京}
\date{西\,历\today}
\maketitle

\begin{abstract}
  本文档将介绍{\scshape JSTML},一种基于{\scshape C}语言构建的(极简易的)文本结构化标记语言;而它的设计目的是,用来写同学录。\par
  其本质上是一个基于下推自动机\footnote{一个很哲学的术语。}的解析器,因设计用途的局限性,不允许出现嵌套等魔法。
  容错模型也较为简陋,而性能则没有进行任何优化(读入输出纯靠栈)。\par
  将先介绍数据结构、语法,而后介绍实现细节、自动化{\scshape Lua}脚本、{\scshape plain-TeX}输出样式文件等信息。
\end{abstract}

\section{一些约定}
\begin{itemize}
  \item 下文中将会用「她」「其」等代词表示{\scshape JSTML}语言的独立解释器,即她的可执行文件。
  \item 将会用小型大写西文字母({\scshape Small Caps})表示脚本语言、宏语言、标记语言等,还用来表示某些特定的操作系统名称。
  \item 使用方全角引号(「」)表示被它们划定的特定字符;使用方括号([])表示可选项描述,不代表实际键入的字符。
  \item 将会使用脚注补充一些多馀信息,且响应国家倡议:正文中使用符合现行语言标准的简化字和两个全角空格宽的缩进。
\end{itemize}

\section{字类型及数据类型、结构}
\subsection{字类型}
在她眼中,所有的输入都属于字(token),及一个或一些字符(character)的集合。
而字又被分为四种类别\footnote{致敬高德纳教授所创{\TeX}的类别码(category code),
由于大部分(如果不是全部的话)标记语言都有类别码的概念、而{\TeX}实际上是宏语言,故特此说明。}:
\begin{description}
  \item[分界符] 如其名,自然是作为两种数据结构的界定出现。其中,「<」被用作表示开始、而「>」表示结束。
  \item[标示符] 标示其中一种数据结构的开始与结束。其中,「*」为开,而「/」为关。
  \item[汉字] 主要的处理对象,也就是这门语言「标记」的东西。由几乎所以不属于其它三类的字符组成。
  \item[空白] 包括空格\footnote{不包括中文的全角空格,其属于汉字类别。}及横向制表符(tab)。
\end{description}\par
\subsection{数据类型}
上述的四种字组合便有了能够被她处理的,合法的唯二的数据类型:狗牌(tag)和八卦(text)。
其中狗牌表示对八卦的一个概述,故理论上应短小而精悍。也因此,她内部分配给狗牌的空间仅有19个字符长度
\footnote{实现使用\lstinline|<uchar.h>|的标准化头文件来支持万国码,所以一个字符是8字节长度。}。
也就是说,如果你往狗牌里硬塞超过19个汉字,会导致分段错误或栈溢出,报错并继续运行
(毕竟它只是狗牌呐)。
而另一种八卦所能容纳的字符就多多了,达8192个字符的长度。
八卦与狗牌一一对应,是对狗牌的展开说明,等等一切合理的用途。
\subsection{数据结构}
在两种数据类型的基础上,又构建了两种数据结构(即,用来组织/表示数据类型的东西):片段和累牍。
一个片段只能出现在一行之中\footnote{实际上,是由于它的末尾被且仅被换行符界定。这由于平台差异会出现事故:
{\scshape Windows}下换行符为\lstinline|<CR><LF>|、而{\scshape Macintosh}和{\scshape Unix}系为\lstinline|<CR>|、
{\scshape Posix}等不明确。故不支持使用{\scshape Windows}系统构建项目。},而累牍则理论上横跨数行。\par
片段的狗牌被使用一对分界符界定,而八卦则被结束分界符和换行符界定,语法如下:
\begin{lstlisting}
[optspace]<[optspace]狗牌[optspace]>[optspace]八卦[carriage return]
\end{lstlisting}
而\lstinline|[optional space]|表示可选的被忽略的空白字类型、\lstinline|[carriage return]|表示换行符(回车)。

\end{document}
