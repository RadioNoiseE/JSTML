%% %%%%%%File jsdvr.tex%%%%%%%
%% %part of the JSTML project%
%% %%the plainTeX style file%%
%% %%%%%author Jing Huang%%%%%
%% %%%%%Copyright(c) 2023%%%%%
%% %%%%%%%%MIT License%%%%%%%%

%! TeX Program = LuaTeX

%% ==> 導言區
\catcode`@=11

% PDF紙面設置
\pagewidth=19cm
\pageheight=24cm

% PDF缺省信息
\pdfextension info {
  /Title    (jsdvr.pdf)
  /Author   (JSTML)
  /Subject  (Structured Text Markup Language)
  /Keywords (Markup)
}

% PDF開啟動作
\pdfextension catalog {
  /PageMode /UseOutlines
} openaction goto page2 {/Fit}

% 使頁面處於紙面正中(用TeX的元語)
\hoffset=-.54cm
\voffset=-.54cm

% 頁面大小(雙欄設置信息、非實際)及相關信息
\hsize=7.3cm
\vsize=40.4cm
\maxdepth=2.6pt

% 修改plain-TeX默認的輸出設定(雙欄)
\output={\doublecolumnout}
\def\doublecolumnout{%
    \splittopskip=\topskip \splitmaxdepth=\maxdepth \dim@=20cm
    \setbox0=\vsplit255 to \dim@
    \setbox2=\vsplit255 to \dim@
    \wd0=\hsize \wd2=\hsize
    \shipout\vbox to 20cm{%
            \offinterlineskip
            \hbox to 15cm{\box0\hfil\box2}}
    \unvbox255\penalty\outputpenalty}

% 中日文預設定
\def\ltj@stdmcfont{SourceHanSerif SC}
\def\ltj@stdgtfont{SourceHanSans SC}
\def\ltj@stdyokojfm{eva/{smpl,nstd,hgp}}
\def\ltj@stdtatejfm{eva/{smpl,nstd,hgp,vert}}

% 載入LuaTeX-ja並初始化字體(全局10pt明朝體)
\input luatexja.sty

% JSTML參數預處理(行長為字寬整數倍)
\newdimen\txtwidth
\newdimen\tagwidth
\txtwidth=6cm
\divide\txtwidth by \zw
\multiply\txtwidth by \zw
\tagwidth=7.3cm
\advance\tagwidth by -\txtwidth

% 設置左虛擬邊距、標記使用\llap置於其間
\leftskip=\tagwidth

% JSTML支持宏
\def\jstml@tag#1{%
    \llap{\hss\vbox{\hrule\hbox{%
          \vrule\kern2.6pt\vbox{%
                \kern2.6pt\tengt{#1}\kern2.6pt}%
          \kern2.6pt\vrule}%
    \hrule}\hskip6.6pt}}
\def\jstml@txt#1{%
    \hrule\kern2.6pt\tenmin{#1}%
    \nobreak\vskip2.6pt\nobreak\hrule\vskip4pt}

\catcode`@=12
%% 導言區 <==

%% ==> 正文區
\msg{修改jsdvr.tex中的程式碼以改變缺省設置給}

% 第一頁「標題」
\begingroup
\hsize=2in
\tenmin
JSTML文本标记语言\\
缺省输出
\endgroup
\vfill\eject

% 其餘實際輸出
\input jsindex.ind

\bye
%% 正文區 <==
